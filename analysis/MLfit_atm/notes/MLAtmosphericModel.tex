%\documentclass{article}           %% ceci est un commentaire (apres le caractere %)
%\usepackage[latin1]{inputenc}     %% adapte le style article aux conventions francophones
%\usepackage[T1]{fontenc}          %% permet d'utiliser les caract�res accentu�s
%\usepackage[dvips]{graphicx}      %% permet d'importer des graphiques au format .EPS (postscript)
%\usepackage{fancybox}		   %% package utiliser pour avoir un encadr� 3D des images
%\usepackage{makeidx}              %% permet de g�n�rer un index automatiquement


\documentclass{article}[12pt]
\usepackage[left=1.5cm,right=1.5cm,top=1.5cm,bottom=2cm]{geometry} % page settings
\usepackage{amsmath} % provides many mathematical environments & tools
\usepackage{indentfirst}
\setlength{\parindent}{24pt}
%\setlength{\baselineskip}{1cm}
%\parskip = \baselineskip
%\parskip = 1cm 
%\setlength{\parskip}{1cm}
%\usepackage[parfill]{parskip}

\setlength{\parindent}{4em}
\setlength{\parskip}{1em}
\renewcommand{\baselinestretch}{1.1}


\usepackage[T1]{fontenc} % Use 8-bit encoding that has 256 glyphs
\usepackage[utf8]{inputenc} % Required for including letters with accents
\usepackage{graphicx} % Required for including images
\graphicspath{{figures/}} % Set the default folder for images


\usepackage{enumitem} % Required for manipulating the whitespace between and within lists
\usepackage{lipsum} % Used for inserting dummy 'Lorem ipsum' text into the template
%\usepackage{subfig} % Required for creating figures with multiple parts (subfigures)
\usepackage{subcaption}
\usepackage{amsmath,amssymb,amsthm} % For including math equations, theorems, symbols, etc
\usepackage[toc]{appendix}

\usepackage{tikz}
\usepackage{pgfplots}




%\usepackage{pst-func}
%\pgfplotsset{compat=newest}

%\directlua{
%  ffi=require("ffi")
%  ffi.cdef[[
%  double jn(int n, double x);
%  double yn(int n, double x);
%  ]]
%}

%\pgfmathdeclarefunction{BesselJ}{2}{%
%  \edef\pgfmathresult{%
%    \directlua{tex.print(ffi.C.jn(\pgfmathfloatvalueof{#1},%\pgfmathfloatvalueof{#2}))}%
%  }%
%}

%\pgfmathdeclarefunction{BesselY}{2}{%
%  \edef\pgfmathresult{%
%    \directlua{tex.print(ffi.C.yn(\pgfmathfloatvalueof{#1},%\pgfmathfloatvalueof{#2}))}%
%  }%
%}





\graphicspath{{figures/}} % Set the default folder for images
\setlength{\parindent}{0mm}
\usepackage{hyperref}

\title{Machine learning Atmospheric Model}     %% \title est une macro, entre { } figure son premier argument
\author{Sylvie Dagoret-Campagne}        %% idem

\makeindex		    %% macro qui permet de g�n�rer l'index
\bibliographystyle{prsty}	  %% le style utilis� pour cr�er la bibliographie
\begin{document}                  %% signale le d�but du document



\maketitle                        %% produire � cet endroit le titre de l'article � partir des informations fournies ci-dessus (title, author)
%\newpage
%\tableofcontents                  %% produire � cet endroit la table des mati�ree				

\section{Formula}	

\fbox{\parbox{0.65\textwidth}{
\begin{eqnarray}
\left. \delta F(\lambda,t)\right|_{ADU} & = & \frac{S_{coll}}{g_{el}} \cdot T^{atm}(\lambda,t) 
\cdot T^{opt}(\lambda)
\cdot T^{grat}(\lambda)
\cdot \epsilon_{CCD}(\lambda) 
\cdot \frac{dN_{\gamma}}{d\lambda}(\lambda) d\lambda \nonumber \\
& = & Cte(\lambda ) \times T^{atm}(\lambda,t) d\lambda \nonumber \\
T^{atm}(\lambda,t) & = & \exp\left( -\tau_{cld}(t)z -\tau_{ray}(\lambda)z
-\tau_{aer}(\lambda,t)z  \right. \nonumber \\
&  - & \left. \sum_{n=O_2,O_3,H_2O} \kappa_{n}(\lambda,t)f_n(z)
\right) \nonumber \\
\tau_{ray}(\lambda)& \approx & \tau_{ray}\cdot g_{ray}(\lambda) \nonumber \\   
\tau_{aer}(\lambda,t)& \approx & \tau_{aer}(t)\cdot g_{aer}(\lambda) \nonumber \\  
\kappa_n(\lambda,t) & \approx & \kappa_n(t) \cdot g_{n}(\lambda) \nonumber \\
g_n(\lambda ), \kappa_n(\lambda) & & {\rm known \,\,\, templates} \nonumber \\
f_n(z) & \approx & z^\gamma \,\, \, , \,\, \frac{1}{2} \le \gamma \le 1 \nonumber \\
& \approx & (1-\gamma) + \gamma z + \cdots \nonumber
\end{eqnarray}
}}


\fbox{\parbox{0.4\textwidth}{
\begin{eqnarray}
\frac{2.3}{2.5}m(\lambda,t) & = & Cte(\lambda) + \nonumber \\
&  & \tau_{ray}(\lambda)\cdot z + \nonumber  \\
&  & \tau_{cld}(t)\cdot z + \nonumber  \\
&  & \tau_{aer}(t)\cdot g_{aer}(\lambda)\cdot z + \nonumber  \\
&  & \sum_{n=O_2,O_3,H_2O} \kappa_{n}(t)g_{n}(\lambda)\cdot f_n(z) \nonumber
\end{eqnarray}
}}

\fbox{\parbox{0.6\textwidth}{
\begin{eqnarray}
t & \rightarrow & i \,\,\,{\rm sample \,\, number}, \,\, 0\le i \le N_{obs} \nonumber \\
\lambda & \rightarrow & j \,\,\,{\rm wavelength\,\, index}, \,\,j < 700 \,\,( 400 nm \le \lambda < 1100 nm) \nonumber \\
m^{\prime i}_j & = & \tau_{cld}^i + g_{aer \, j} \cdot \tau_{aer}^i +
\sum_{n=O_3,H_2O} g_{n\,\,j}\cdot h_n(z) \cdot \kappa^i_{n}
\nonumber \\
m^{\prime\,i}_j  & = & \frac{\frac{2.3}{2.5}m^i_j - Cte_j}{z} - \tau_{ray\,j}- 
g_{O_2\,\,j}\cdot h_{O_2}(z) \cdot \kappa_{O_2} \nonumber \\
h_n(z) & = & \frac{f_n(z)}{z} \nonumber
\end{eqnarray}
}}

\fbox{\parbox{\textwidth}{
\begin{itemize}
\item apriori known constants :
\begin{itemize}
\item $Cte_j$ known from instrument calibration
\item $\tau_{ray\,j}$ known from atmospheric model , ex libradtran, (analitic formula known)
\item $g_{aer \, j}$ known from atmospheric model , ex libradtran,
\item $g_{O_2\,\,j}$ known from atmospheric model , ex libradtran,
\item $g_{O_3\,\,j},g_{H_2O\,\,j}$ known from atmospheric model , ex libradtran,
\item $f_n(z)$ known from atmospheric model , ex libradtran
\end{itemize}
\item for each spectrum observation $i$ :
\begin{itemize}
\item $\approx 700$ $j$-observations $m^i_j$ (or $m^{\prime i}_j$),
\item only 4 unknown variables to estimate: $\tau_{cld}^i$,$\tau_{aer}^i$, $\kappa^i_{O_3}$,$\kappa^i_{H2O}$,
\item linear equations between observed variables and variables to estimate.
\end{itemize}
\end{itemize}
}}


\fbox{\parbox{0.55\textwidth}{
\begin{eqnarray}
\tau_{cld}^i & =&  \beta_0^{cld} + \sum_{j=1}^{N_j} \beta_j^{cld} \cdot m_j^i \nonumber \\
\tau_{aer}^i & =&  \beta_0^{aer} + \sum_{j=1}^{N_j} \beta_j^{aer} \cdot m_j^i \nonumber \\
\kappa_{O_3}^i & =&  \beta_0^{O_3} + \sum_{j=1}^{N_j} \beta_j^{O_3} \cdot m_j^i \nonumber \\
\kappa_{H_2O}^i & =&  \beta_0^{H2_0} + \sum_{j=1}^{N_j} \beta_j^{H_20} \cdot m_j^i \nonumber
\end{eqnarray}
\begin{itemize}
\item number of coefficients $\beta_j^n$  to estimate : $4 \times(N_j+1) \approx 2800 $
\end{itemize}
}}

\fbox{\parbox{0.5\textwidth}{
\begin{eqnarray}
f_z(z) & = & z^\gamma = (1+\delta z)^\gamma = 1+ \gamma \delta z - \frac{1}{2}(\gamma-\gamma^2) (\delta z)^2 \nonumber \\
g_z(z) & = & z^{-\gamma} = 1 - \gamma \delta z + \frac{1}{2}(\gamma^2+\gamma) (\delta z)^2 \nonumber \\
z & = & 1 + \delta z \nonumber
\end{eqnarray}
}}

\fbox{\parbox{\textwidth}{
\begin{itemize}
\item add interaction terms $g_z(z^i)\cdot m^i_j$:
\begin{itemize}
\item  $\delta z^i \cdot m^i_j$,
\item  $(\delta z^i)^2 \cdot m^i_j$,
\end{itemize} 
\end{itemize}
\begin{eqnarray}
\tau_{cld}^i & =&  \beta_0^{cld 0} + \beta_0^{cld 1}\cdot \delta z^i + \beta_0^{cld 2}\cdot(\delta z^i)^2  + \sum_{j=1}^{N_j} \left( \beta_j^{cld 0} \cdot m_j^i + \beta_j^{cld 1} \cdot m_j^i\cdot \delta z^i + \beta_j^{cld 2} \cdot m_j^i\cdot(\delta z^i)^2  \right) \nonumber \\
\tau_{aer}^i & =&  \beta_0^{aer 0} + \beta_0^{aer 1}\cdot \delta z^i + \beta_0^{aer 2}\cdot(\delta z^i)^2 + \sum_{j=1}^{N_j} \left( \beta_j^{aer 0} \cdot m_j^i \beta_j^{aer 1} \cdot m_j^i\cdot \delta z^i + \beta_j^{aer 2} \cdot m_j^i\cdot(\delta z^i)^2   \right) \nonumber \\
\kappa_{O_3}^i & =&  \beta_0^{O_3 0} + \beta_0^{O_3 1}\cdot \delta z^i + \beta_0^{O_3 2}\cdot(\delta z^i)^2  + \sum_{j=1}^{N_j} \left( \beta_j^{O_3 0} \cdot m_j^i + \beta_j^{O_3 1} \cdot m_j^i \cdot \delta z^i + \beta_j^{O_3 2} \cdot m_j^i\cdot(\delta z^i)^2  \right) \nonumber \\
\kappa_{H_2O}^i & =&  \beta_0^{H_2O 0} + \beta_0^{H_2O 1}\cdot \delta z^i + \beta_0^{H_2O 2}\cdot(\delta z^i)^2 + \sum_{j=1}^{N_j} \left( \beta_j^{H_2O 0} \cdot m_j^i + \beta_j^{H_2O 1} \cdot m_j^i\cdot \delta z^i + \beta_j^{H_2O 2} \cdot m_j^i\cdot(\delta z^i)^2 \right) \nonumber
\end{eqnarray}

}}


\fbox{\parbox{\textwidth}{
\begin{eqnarray}
\tau_{cld}^i & =&  \beta_0^{cld}/z^i + \sum_{j=1}^{N_j} \beta_j^{cld} \cdot m_j^i/z^i \nonumber \\
\tau_{aer}^i & =&  \beta_0^{aer}/z^i + \sum_{j=1}^{N_j} \beta_j^{aer} \cdot m_j^i/z^i \nonumber \\
\kappa_{O_3}^i & =&  \beta_0^{O_3 0} + \beta_0^{O_3 1}\cdot \delta z^i + \beta_0^{O_3 2}\cdot(\delta z^i)^2  + \sum_{j=1}^{N_j} \left( \beta_j^{O_3 0} \cdot m_j^i + \beta_j^{O_3 1} \cdot m_j^i \cdot \delta z^i + \beta_j^{O_3 2} \cdot m_j^i\cdot(\delta z^i)^2  \right) \nonumber \\
\kappa_{H_2O}^i & =&  \beta_0^{H_2O 0} + \beta_0^{H_2O 1}\cdot \delta z^i + \beta_0^{H_2O 2}\cdot(\delta z^i)^2 + \sum_{j=1}^{N_j} \left( \beta_j^{H_2O 0} \cdot m_j^i + \beta_j^{H_2O 1} \cdot m_j^i\cdot \delta z^i + \beta_j^{H_2O 2} \cdot m_j^i\cdot(\delta z^i)^2 \right) \nonumber
\end{eqnarray}
}}


\newpage
\begin{thebibliography}{99}

\bibitem{ISL}{G. James 2015} \\
An introduction to Statistical learning with application in R
\url{http://faculty.marshall.usc.edu/gareth-james/ISL/}

\bibitem{ESL}{T. Hastie 2013} \\
The elements of statistical learning
\url{https://web.stanford.edu/~hastie/Papers/ESLII.pdf}


\end{thebibliography}



\end{document}